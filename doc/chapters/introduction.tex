\chapter{Introduction}
\label{chapter:introduction}

% motivation
Melanoma is a dangerous type of skin cancer which in 2012 occurred in 232000 people, and in 2015 there were 3.1 million people with the disease that resulted in over 59000 deaths. According to the Centers for Disease Control and Prevention, the rates of new melanomas have doubled over the last three decades and will continue to double.

% definition
Melanoma is a cancer that is developed from melanocytes (i.e. cells that produce the melanin skin pigment). This abnormal growth of tissue generally occurs in skin, but can also manifest itself in the mouth, intestines or eyes. The most common cause of melanoma is the exposure to ultraviolet light (e.g. sunlight, tanning devices), thus it can be prevented by frequent use of sunscreen and avoiding long exposure to the sun. The clinical diagnosis is confirmed with a skin biopsy and, if it hasn't spread, treatment is usually surgical excision.

Since the skin lesions occur on the surface of the skin, melanoma can easily be detected early through visual inspection by a physician with the use of dermoscopy techniques that allow a better look at the pigmented lesions. Dermoscopy is an imaging technique that works by removing the surface reflection of the skin which enables the visualization of enhanced levels of skin. More recently, computerized digital dermoscopy has made it relatively trivial to get high resolution imaging that can be used to get second opinions remotely or even a computer-assisted diagnosis\cite{dermoscopy}. Specifically, advances in deep learning algorithms and computer hardware has made classification by a machine learning algorithm a viable and reliable technique for a diagnosis.
