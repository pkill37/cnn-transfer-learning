\chapter{Conclusion}
\label{chapter:conclusion}

In this section we present conclusions, final remarks, and point to directions for future work.

\section{Discussion}

In conclusion:

\begin{itemize}
    \item In transfer learning, the learning rate is very important, especially when extracting lower layers which empirically seem to be more sensitive to overshooting updates when compared to higher layers. Alternatively an adaptive optimizer can solve the same fundamental issue;
    \item Designing a custom \ac{CNN} and training it from scratch is difficult because it requires reasoning and setting many hyperparameters simultaneously, and cross-validating a wide range of values for so many hyperparameters is computationally expensive. Transfer learning emerges as the clear solution to a lot of problems where data is scarce, especially in computer vision where the availability of pre-trained models is quite diverse.
\end{itemize}

\section{Future Work}

A few lines of future work are possible:

\begin{itemize}
    \item The VGG16 architecture, although very capable given its age, is not state-of-the-art anymore and attention has long shifted to residual networks. In such complex networks it is not so clear how to best take advantage of the parameters in the layers of the pre-trained model, and the kind of exhaustive study that can easily be done for VGG16 is not so feasible for such networks because the number of layers is in the hundreds which quickly originates a combinatorial explosion of all the possible configurations for extracting and freezing parameters. In that sense, it would be interesting to see a similar study for more recent architectures, which of course requires a very deep understanding of their inner workings;
    \item Systematic and thorough study of optimizers and learning rates in transfer learning applied to skin lesion classification, perhaps even developing optimizers or learning rate schedules specific to transfer learning.
\end{itemize}
