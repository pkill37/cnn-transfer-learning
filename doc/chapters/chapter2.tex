\chapter{Transfer Learning in Dermatology}
\label{chapter:sota}

% TODO: rewrite with transfer learning (not deep learning) in mind

Deep learning is providing exciting solutions for medical image analysis problems and they are seen as a key method for future applications. The initial impact of deep learning for medical imaging was revealed through a special issue published in the IEEE Transactions on Medical Imaging (Greenspan, Ginneken and Summers, 2016). The surveys by Hu et al., (2018) and Litjens et al. (2017) contribute also to a clear understanding of the principles and methods of neural network and deep learning concepts, showing how these algorithms are being applied to medical image in a wide variety of application areas.

The successes of deep learning architectures such as \ac{DNN}, \ac{DBN}, \ac{RNN} are now well reported in the areas of computer vision, speech recognition, natural language processing and gaming. A comprehensive and up to date approach to deep learning can be found elsewhere (Goodfellow, Bengio and Courville, 2016; LeCun, Bengio and Hinton, 2015). This section provides a brief presentation of what deep learning is, the computational advantages and the current applications in dermoscopic image analysis.

\section{CNN Classifiers}
Most skin lesion classifiers use \ac{CNN} \cite{brinker2018}. The success of this method is due to.

\section{Type 2}
Lorem ipsum

\section{Type 3}
Lorem ipsum
